\documentclass[12pt, a4paper]{article}
\usepackage[a4paper,
left=15mm,
right=15mm,
top=15mm,
bottom = 15mm]{geometry}
\usepackage{amsmath}
\usepackage{graphicx}
\usepackage{algorithm}
\usepackage{algorithmic}
\usepackage{multicol}
\usepackage{amsthm}
\usepackage{bm}
\usepackage{fancyhdr}
\usepackage{amssymb}
\usepackage{pifont}
\usepackage{array}
\usepackage[hidelinks]{hyperref}
\numberwithin{figure}{section}
\usepackage[dvipsnames]{xcolor}

\newcommand{\cmark}{\ding{51}}%
\newcommand{\xmark}{\ding{55}}%

\newcolumntype{P}[1]{>{\centering\arraybackslash}p{#1}}  %per centrare le colonne nelle tabelle
\newcolumntype{M}[1]{>{\centering\arraybackslash}m{#1}} %per centrare le righe nelle tabelle
\renewcommand{\arraystretch}{1.1} 

\newcommand\tab[1][1cm]{\hspace*{#1}}
\renewcommand{\labelitemii}{$\star$}

\begin{document}


\begin{titlepage}
	\begin{center}
		\vspace*{1cm}

		\Huge{Tesina\\Human Computer Interaction}
		\vspace{1.5cm}
		\Huge
		\textbf{\\CiakTime}
		\vspace{1.5cm}

		\Large
		Authors:\\
		\textbf{Mauro Ficorella 1941639}\\
		\textbf{Martina Turbessi 1944497}\\
		\textbf{Valentina Sisti 1952657}\\
		\vspace{0.5cm}

		\vfill

		\includegraphics[width=0.4\textwidth]{Images/Logo.jpg}

		\vfill

		\vspace{0.8cm}

		\Large
		Sapienza\\
		July 2021
	\end{center}
\end{titlepage}


\tableofcontents{}
\thispagestyle{empty}


% INTRODUZIONE --------------------------------------------------------------------

\newpage

\setcounter{page}{1}

\section{Introduction}

Since nowadays there are a lot of movies out
both in theatres and in streaming platforms, we had the idea to realize CiakTime, so that cinema lovers can satisfy their
needs to stay updated with the latest movies and keep track of them.
Also they may want to know on which streaming platforms they can found the movies.
Finally they could have the necessity to interact with other cinema lovers about their
favourite movies or share their opinion about the movie with the community through reviews.
\\\\
For these reasons, our app offers a lot of functionalities.
The user has the possibility to keep track of already watched movies, movies to watch and favourite movies;
moreover he can search for movies by title, also filtering results, search for actors and movie directors,
look for upcoming movies and popular movies and actors.
Regarding the movies, he can read information about plot, cast, year of release, duration, genre,
movie director and streaming platform on which the movie is available; in addition, he can
review and rate movies, comment and like reviews made by other users.
Finally, regarding movie directors and actors, the user can read their biography and take a look to their
filmography.
\\\\
In order to involve as much users as possible, we decided to make our app available for both iOS and Android devices.





% Requirement analysis -------------------------------------------------------------------------

\newpage

\section{Requirement analysis}

To realize our application, we followed the \textit{User Centered Design} (UCD)  approach, which intends
to ensure that the user is at the center during the design process in order to realize products
that meet usability requirements.\\
Since humans become the center of our interest, the system is created according to their perspective.
So we need to involve user throughout the creating process in order to learn as much things as possible
about our type of product and the final customers.
To do that, we start collecting some information through competitors analysis, user analysis and
questionnaires analysis.

% Competitor  analysis -------------------------------------------------------------------------

\subsection{Competitor analysis}

Since a system needs to compare itself with what’s already on the market, both in the pros and in
the cons, one of the first requirments analysis to be done is \textit{Competitor analysis}.
In this way, we can add something that is new, innovative and valuable to the user.\\\\
We found two main competitors for our application: \textbf{IMDb} and \textbf{Cinemaniac}.
\paragraph{IMDb}\mbox{}\\\\
\includegraphics[width=0.4\textwidth]{Images/IMDb.png}\\
IMDb is the world's most popular and authoritative source for movie, TV, and celebrity information. This app has a
huge fanbase and a limitless cinema database.
On this app the user can watch trailers, get showtimes, and buy tickets for upcoming films. He can rate and review shows he has seen
and track what he wants to watch using his Watchlist, and he can also get suggestions regarding movies based on it.\\
However, we have identified few weaknesses, such as the impossibility to exchange opinions between users, to keep track
of already watched movies and to save favourite movies in a list; it is also not very intuitive to retrieve movies specific
information due to the high number of functionality offered by the application.

\paragraph{Cinemaniac}\mbox{}\\\\
\includegraphics[width=0.4\textwidth]{Images/Cinemaniac.png}\\
Cinemaniac is an app on which the user can search for a movie and add it to the “Movies to watch”, “Watched movies” or favourite list.
He can see all the relevant details for any movie and he can leave his own personal grade.
The user can find suggestions on the most popular and top rated movies.
Moreover, he can find a specific list relative to currently projected movies and upcoming titles.\\
Also here we have identified some weaknesses, like the fact that the interface is not so user friendly, there is no user interaction,
there are no information about streaming platforms; moreover the search about movies is not so intuitive and there are
in-app purchases required to remove advertisements and unlock some functionalities.\\\\
In the following table we summarize the comparison between our app and the competitors:

\begin{center}
	\begin{tabular}{ |p{5cm}|P{3cm}|P{3cm}|P{3cm}|  }
		\cline{2-4}
		\multicolumn{1}{c|}{}
		                               & \textbf{CiakTime} & \textbf{IMDb} & \textbf{Cinemaniac} \\
		\hline
		\textbf{User profile}          & \checkmark        & \checkmark    & \xmark              \\
		\hline
		\textbf{Search}                & \checkmark        & \checkmark    & \checkmark          \\
		\hline
		\textbf{Movie info}            & \checkmark        & \checkmark    & \checkmark          \\
		\hline
		\textbf{Streaming platform}    & \checkmark        & \checkmark    & \xmark              \\
		\hline
		\textbf{Upcoming movies}       & \checkmark        & \checkmark    & \checkmark          \\
		\hline
		\textbf{Watch history}         & \checkmark        & \xmark        & \checkmark          \\
		\hline
		\textbf{Watch list}            & \checkmark        & \checkmark    & \checkmark          \\
		\hline
		\textbf{Favourite movies}      & \checkmark        & \xmark        & \checkmark          \\
		\hline
		\textbf{Review movies}         & \checkmark        & \checkmark    & \checkmark          \\
		\hline
		\textbf{Rate movies}           & \checkmark        & \checkmark    & \checkmark          \\
		\hline
		\textbf{Comment other reviews} & \checkmark        & \xmark        & \xmark              \\
		\hline
		\textbf{Like other reviews}    & \checkmark        & \checkmark    & \xmark              \\
		\hline
		\textbf{No ads}                & \checkmark        & \checkmark    & \xmark              \\
		\hline
	\end{tabular}
\end{center}
\mbox{}\\
% User analysis -------------------------------------------------------------------------

\subsection{User analysis}
In this section we want to analyze the possible users for our application.
In particular we describe the User Profile, which is a detailed description of our users'
attributes, the Personas, which are fictional individuals created to describe the typical user
based on the user profile and the Scenarios, which are stories that describe how a particular
persona completes a task or behaves in a given situation.

\subsubsection{User Profile}
In the following table we show the general description of our target users and their characteristics.
\begin{center}
	\begin{tabular}{ |p{3cm}|P{7cm}|  }
		\hline
		\textbf{Age}        & 18-50 years                 \\
		\hline
		\textbf{Gender}     & male/female                 \\
		\hline
		\textbf{Profession} & Any                         \\
		\hline
		\textbf{Education}  & Any                         \\
		\hline
		\textbf{Location}   & Any                         \\
		\hline
		\textbf{Tecnology}  & Basic smartphone experience \\
		\hline
		\textbf{Passions}   & Cinema, movies              \\
		\hline
	\end{tabular}
\end{center}

\subsubsection{Persona 1 - Vittoria}

\begin{minipage}{0.3\textwidth}
	\includegraphics[width=1\textwidth]{images/vittoria.png}
\end{minipage}
\hspace{0.02\linewidth}
\begin{minipage}{0.65\textwidth}
	\textbf{Age:} 25 years-old \\
	\textbf{Gender:} Female\\
	\textbf{Profession:} Student\\
	\textbf{Education:} University student\\
	\textbf{Location:} Rome, Italy\\
	\textbf{Tecnology:} Mid level\\
	\textbf{Passions:} Watching movies and tv-series on streaming platforms \\
\end{minipage}

\paragraph{Persona}\mbox{}\\
Vittoria is 25 years-old and comes from Rome. She is a university student and in the free
time her main hobby is watching movies and tv-series on her favourite streaming platforms.
During her study breaks she likes to keep in touch with her friends on various social apps.

\paragraph{Scenario}\mbox{}\\
Vittoria has just terminated an intense study session and now she only wants to
relax watching a movie. She decides to call her best friend to spend the evening together.
Once she arrives, in order to choose which movie to watch, they both open the app to
compare their watchlists. After a while they realize that they both have “La La Land“ in their
watchlists and so they decide to watch it together.
At the end of the evening they both check it as “watched“ in their app.

\subsubsection{Persona 2 - Emanuele}

\begin{minipage}{0.3\textwidth}
	\includegraphics[width=1\textwidth]{images/emanuele.png}
\end{minipage}
\hspace{0.02\linewidth}
\begin{minipage}{0.65\textwidth}
	\textbf{Age:} 33 years-old\\
	\textbf{Gender:} Male\\
	\textbf{Profession:} Programmer\\
	\textbf{Education:} Degree\\
	\textbf{Location:} Torino, Italy\\
	\textbf{Tecnology:} High level\\
	\textbf{Passions:} Action movies, technology\\
\end{minipage}

\paragraph{Persona}\mbox{}\\
Emanuele is 33 years-old and comes from Torino. He is a programmer and he likes
very much going to the cinema with his girlfriend.
As a programmer, he is addicted of technology in general, and more specifically of mobile
devices; moreover he is a very organized guy, and so he likes to keep under control
everything he does in his life using mobile apps.

\paragraph{Scenario}\mbox{}\\
It is an afternoon autumn day. Emanuele and his girlfriend would have liked to go
out for a walk, but since it’s raining, they don’t know what to do. So, Emanuele opens
the app in search of a popular movie. In this list he founds that is just
available in the cinemas a new action movie with his favourite actor Vin Diesel; since also his
girlfriend likes action movies, they decide to go to the cinema to watch it and spend
a good afternoon together.


% Questionnaire  analysis -------------------------------------------------------------------------

\subsection{Questionnaire analysis}

Questionnaires are a useful method to investigate user needs, expectations, perspectives, priorities and preferences.
They are useful in user requirement but also in evaluation phase to investigate user satisfaction, user attitudes and opinions, relevance of collections and services to user needs, trends.
We designed the questionnaire in such a way that each question was clearly written, in order to not lead the user to a specific answer and to
always make them feel confortable while answering.
Below we present the questionnaire results used to better understand the target of potential users in order to have a better refinement of
some aspects of our application.
More precisely, we reached 151 people, and so we had a good number of answers, statistically speaking.\\

\paragraph{What's your age?}\mbox{}\\\\
\includegraphics[width=0.7\textwidth]{Images/age.png}\\

\paragraph{What's your gender?}\mbox{}\\\\
\includegraphics[width=0.7\textwidth]{Images/gender.png}\\

\paragraph{What's your educational level?}\mbox{}\\\\
\includegraphics[width=0.7\textwidth]{Images/education.png}\\

\paragraph{How frequently do you use your smartphone on average?}\mbox{}\\\\
\includegraphics[width=0.7\textwidth]{Images/timeAtPhone.png}\\

\paragraph{How many times do you go to the cinema on average? (Before pandemic)}\mbox{}\\\\
\includegraphics[width=0.7\textwidth]{Images/timeAtCinema.png}\\

\paragraph{How many times do you watch movies on streaming platforms/tv on average?}\mbox{}\\\\
\includegraphics[width=0.7\textwidth]{Images/timeStreaming.png}\\

\paragraph{Which streaming platforms do you know?}\mbox{}\\\\
\includegraphics[width=1\textwidth]{Images/streamingPlatform.png}\\\\

\paragraph{Which streaming platforms do you use?}\mbox{}\\\\
\includegraphics[width=1\textwidth]{Images/streamingPlatformUsed.png}\\

\paragraph{Do you use any movies related app?}\mbox{}\\\\
\includegraphics[width=0.7\textwidth]{Images/app.png}\\

\paragraph{If you use any movies related app, which one?}\mbox{}\\\\
\includegraphics[width=1\textwidth]{images/appUsed.png}\\\\\\\\

\paragraph{If you don't use any movies related app, how much would you be interest in using one?}\mbox{}\\\\
\includegraphics[width=1\textwidth]{images/interesting.png}\\

\paragraph{How much do you think the following features are important in such an app?}\mbox{}\\\\
\includegraphics[width=1\textwidth]{Images/features.png}\\

\subsubsection{Conclusions}
After having analyzed the results obtained from the questionnaires, we have formalized the following conclusions:
\begin{itemize}
	\item {Regarding ages, we noticed that there is no predominant range, but they
	      are more or less equally distributed between 18 and 50, and the majority
	      of them are women.}
	\item {The majority of them uses smartphone more than 3 hours per day.}
	\item {Since we noticed that the majority of people rarely goes to cinema and
	      conversely watches very often movies on streaming platform, we decided to focus
	      our app on this feature.}
	\item {Moreover, given the fact that a very high number of people does not use
	      a movie related app and that the majority of them would be interested in doing
	      this, we thought that the idea of such an app would be very appreciated.}
	\item {Finally, from the last question, emerge the most wanted features such
	      as have informations about movies, have the possibility to add movies to lists
	      and have informations about upcoming movies, and so we decided to focused on them.}
\end{itemize}



% Task analysis: HTA and STN -------------------------------------------------------------------------

\newpage

\section{Task analysis: HTA and STN}

In this section we are going to show HTA and STN in order to formalize the main task of
our application and to analyze and describe how users can reach their goals.\\
\textit{Hierarchical Task Analysis} (HTA) is a task description methodology that is used to produce a
complete description of tasks in a hierarchical structure of goals, sub-goals, operations and plans
in order to have a complete representation of the action.\\
Instead, a \textit{State Transition Network} (STN) represents a dialogue between the user and the system,
in which the system could support the tasks that the customer has to execute.
It describes which are the available actions at a certain point, and the consequent
state that the system will reach.\\\\
The main tasks that the user can do in our application are:
\begin{itemize}
	\item Login into the app.
	\item Search for a movie, an actor or movie director.
	\item Add a movie into three lists: watchlist, movie already watched list and favourite movie list.
	\item Make a review regarding a movie.
	\item Interact with other users by commenting a review written by another user.
\end{itemize}
\hbox{}
\subsection{Login}
The user can login into the app if registered, either through online platforms (Google, Facebook)
or using his personal email and password, in order to save in the cloud all his data regarding the app.
From now on, in the following HTAs, we assume that the user is logged in.\\

\paragraph{HTA - Login}\mbox{}\\
\begin{figure}[H]
	\centering
	\includegraphics[width=1\textwidth]{images/Login HTA.png}\\
\end{figure}
\mbox{}\\\\
\paragraph{STN - Login}\mbox{}\\
\begin{figure}[H]
	\centering
	\includegraphics[width=0.9\textwidth]{images/LoginSTN.png}\\
\end{figure}

\mbox{}\\

\subsection{Search for a movie or a person}
The user searches for the movie or the person he wants to know information about.
He can do search by title, actor or movie director and, when the results are shown,
he selects the result he is interested in.\\

\paragraph{HTA - Search for a movie or a person}\mbox{}\\
\begin{figure}[H]
	\centering
	\includegraphics[width=0.9\textwidth]{images/Search HTA.png}\\
\end{figure}
\mbox{}\\
\paragraph{STN - Search for a movie or a person}\mbox{}\\
\begin{figure}[H]
	\centering
	\includegraphics[width=1\textwidth]{images/SearchSTN.png}\\
\end{figure}

\subsection{Add movie to a list}
The user can add a movie into a list: he can choose between a watchlist that contains movies to watch
in the future, a list containing movies already watched and a list containing favourite movies.
We are assuming that the user has already selected a movie.

\paragraph{HTA - Add movie to a list}\mbox{}\\
\begin{figure}[H]
	\centering
	\includegraphics[width=0.6\textwidth]{images/Add movie HTA.png}\\
\end{figure}

\paragraph{STN - Add movie to a list}\mbox{}\\
\begin{figure}[H]
	\centering
	\includegraphics[width=1\textwidth]{images/AddMovieSTN.png}\\
\end{figure}

\mbox{}\\

\subsection{Make a review}
The user can make a review regarding a movie. In particular, once he selected a movie,
he can go to the review page either from the movie page or from the page containing all reviews.\\

\paragraph{HTA - Make a review}\mbox{}\\
\begin{figure}[H]
	\centering
	\includegraphics[width=1\textwidth]{images/Make a review HTA.png}\\
\end{figure}
\mbox{}\\\\\\
\paragraph{STN - Make a review}\mbox{}\\
\begin{figure}[H]
	\centering
	\includegraphics[width=1\textwidth]{images/MakeAReviewSTN.png}\\
\end{figure}


\subsection{Comment other users’ reviews }
The user can comment a review written by another user. In particular he can go to the page containing
all the reviews, choose one of them and comment it.\\

\paragraph{HTA - Comment other users’ reviews}\mbox{}\\
\begin{figure}[H]
	\centering
	\includegraphics[width=1\textwidth]{images/Comments HTA.png}\\
\end{figure}
\mbox{}\\
\paragraph{STN - Comment other users’ reviews}\mbox{}\\
\begin{figure}[H]
	\centering
	\includegraphics[width=1\textwidth]{images/CommentsSTN.png}\\
\end{figure}

% Mockup and prototype 0 -------------------------------------------------------------------------

\newpage

\section{Prototype 0: mockups}
In this section we are going to present our first prototype realized through mockups
in Balsamiq Wireframes. In the first prototype we created the skeleton of the application
by modeling the different screens and the interaction between them.\\\\
The main functionalities of this prototype are:
\begin{itemize}
	\item Login and registration of an user.
	\item Search for a movie or a person.
	\item Add a movie to a list.
	\item Review a movie.
	\item Comment other reviews.
\end{itemize}

\paragraph{Login and Registration}
\mbox{}
\begin{figure}[H]
	\centering
	\includegraphics[width=0.5\textwidth]{images/mockups/signInSignUp.png}\\
	\caption{Login and Registration pages}
\end{figure}
\begin{figure}[H]
	\centering
	\includegraphics[width=0.5\textwidth]{images/mockups/loginSocial.png}\\
	\caption{Login with Google and Facebook}
\end{figure}
\paragraph{Search for a movie or a person}
\mbox{}\\\\
The user can do the search by tapping on the search bar and, if he wants, he can apply filters by tapping
on the proper icon.\\
\begin{figure}[H]
	\centering
	\includegraphics[width=1\textwidth]{images/mockups/search2.png}\\
	\caption{Searching process with optional filters}
\end{figure}
\mbox{}\\
\noindent
After having searched for a movie or a person, the user can tap on it and the application, according
to his choice, will show one of the following screens:
\begin{center}
	\begin{minipage}{0.4\textwidth}
		\begin{figure}[H]
			\centering
			\includegraphics[width=0.5\textwidth]{images/mockups/Movie page.png}\\
			\caption{Movie page}
		\end{figure}
	\end{minipage}
	\hspace{0.01\linewidth}
	\begin{minipage}{0.4\textwidth}
		\begin{figure}[H]
			\centering
			\includegraphics[width=0.5\textwidth]{images/mockups/People page.png}\\
			\caption{Person page}
		\end{figure}
	\end{minipage}
\end{center}
\mbox{}\\\\\\\\\\\\\\\\\\\\
\paragraph{Add a movie to a list}
\mbox{}\\\\
The user can add a movie to a list in the following ways: by tapping on the "+" button to add the movie to
"Watchlist" or "Already watched" and by tapping on the heart icon to add the movie to "Favourite movies".
\mbox{}
\begin{figure}[H]
	\centering
	\includegraphics[width=0.5\textwidth]{images/mockups/addMovie.png}\\
	\caption{Add a movie to a list process}
\end{figure}
\noindent
The user can reach the lists where he added the movie from the user profile page,
in order to see all the movies added in watchlist, already watched list and favourite movies list.\\
\begin{figure}[H]
	\centering
	\includegraphics[width=0.21\textwidth]{images/mockups/User profile.png}\\
	\caption{User profile page}
\end{figure}
\begin{center}
	\begin{minipage}{0.31\textwidth}
		\begin{figure}[H]
			\centering
			\includegraphics[width=0.68\textwidth]{images/mockups/Watchlist.png}\\
			\caption{Watchlist}
		\end{figure}
	\end{minipage}
	\hspace{0.01\linewidth}
	\begin{minipage}{0.33\textwidth}
		\begin{figure}[H]
			\centering
			\includegraphics[width=0.65\textwidth]{images/mockups/Already watched.png}\\
			\caption{Movie already watched}
		\end{figure}
	\end{minipage}
	\hspace{0.01\linewidth}
	\begin{minipage}{0.31\textwidth}
		\begin{figure}[H]
			\centering
			\includegraphics[width=0.68\textwidth]{images/mockups/Favourite.png}\\
			\caption{Favourite movie}
		\end{figure}
	\end{minipage}
\end{center}
\mbox{}\\
\paragraph{Review a movie}
\mbox{}\\\\
The user can reach the review form in two ways:
\begin{itemize}
	\item from the movie page;
	\item from the reviews page.
\end{itemize}
As we can see in the following figure:
\begin{figure}[H]
	\centering
	\includegraphics[width=0.9\textwidth]{images/mockups/review.png}\\
	\caption{How to make a review}
\end{figure}
\mbox{}\\\\
\paragraph{Comment other reviews}
\mbox{}\\\\
Tapping on "Comment" button, the user can see the comments of the other users and write a comment.
He can also tap on "Like" button in order to leave positive feedback regarding a certain review.
\begin{figure}[H]
	\centering
	\includegraphics[width=0.6\textwidth]{images/mockups/comment.png}\\
	\caption{Comment other reviews process}
\end{figure}
\paragraph{User profile settings}
\mbox{}\\\\
Moreover, from the settings page, the user can change some profile settings like image profile, username and
password. Also, if the user has not done it before, he can link his profile with his Google or Facebook account.
Finally, here, the user can logout from the app.\\
These are not main functionalities, but we thought that they can be useful in order to support the user.
\begin{figure}[H]
	\centering
	\includegraphics[width=0.6\textwidth]{images/mockups/userSettings.png}\\
	\caption{User profile and user profile settings pages}
\end{figure}
\begin{center}
	\begin{minipage}{0.3\textwidth}
		\begin{figure}[H]
			\centering
			\includegraphics[width=0.8\textwidth]{images/mockups/User setting username.png}\\
			\caption{Change username}
		\end{figure}
	\end{minipage}
	\hspace{0.1\linewidth}
	\begin{minipage}{0.3\textwidth}
		\begin{figure}[H]
			\centering
			\includegraphics[width=0.8\textwidth]{images/mockups/User setting password.png}\\
			\caption{Change password}
		\end{figure}
	\end{minipage}
\end{center}
\begin{center}
	\begin{minipage}{0.3\textwidth}
		\begin{figure}[H]
			\centering
			\includegraphics[width=0.8\textwidth]{images/mockups/User setting google.png}\\
			\caption{Connect google account}
		\end{figure}
	\end{minipage}
	\hspace{0.1\linewidth}
	\begin{minipage}{0.3\textwidth}
		\begin{figure}[H]
			\centering
			\includegraphics[width=0.8\textwidth]{images/mockups/User setting facebook.png}\\
			\caption{Connect facebook account}
		\end{figure}
	\end{minipage}
\end{center}

% Expert Based Evaluation -------------------------------------------------------------------------

\newpage

\section{Expert Based Evaluation}

The expert evaluation is based on our first prototype, the \textit{mockups}. This is a very useful method because
it allows us to detects problems in early stage of the development in order to avoid them in the final implementation.\\\\
We submitted our \textit{mockups} to professor Valeria Mirabella that performed two different types of expert based evaluation:
\textbf{\textit{Heuristic Evaluation}} and \textbf{\textit{Cognitive Walkthrough}}.

\subsection{Heuristic Evaluation}

Heuristic Evaluation is a method used to evaluate if the system follows general usability criteria. Its main goal is to identify any problem associated with the design of user interfaces.
The expert should check if the system is consistent and, if a usability problem occurs, evaluates if it is a major problem,
a minor problem or just something that could be left as it is.
There are various severity levels, and they are assigned in order to make the evaluation.\\
The Heuristic Evaluation used is based on the Jakob Nielsen’s 10 Usability Heuristics:
\begin{enumerate}
	\item {\textit{Visibility of system status}: the system should always keep users informed about
	      what is going on, through appropriate feedback within reasonable time.}
	\item {\textit{Match between system and the real world}: the system should speak the users’ language,
	      with words, phrases and concepts familiar to the user, rather than system-oriented terms. Follow
	      real-world conventions, making information appear in a natural and logical order.}
	\item {\textit{User control and freedom}: users often choose system functions by mistake and will
	      need a clearly marked ”emergency exit” to leave the unwanted state without having to go through an
	      extended dialogue. Support undo and redo.}
	\item {\textit{Consistency and standards}: users should not have to wonder whether different words,
	      situations, or actions mean the same thing. Follow platform conventions.}
	\item {\textit{Error prevention}: even better than good error messages is a careful design
	      which prevents a problem from occurring in the first place.}
	\item {\textit{Recognition rather than recall}: make objects, actions, and options visible.
	      The user should not have to remember information from one part of the dialogue to another.
	      Instructions for use of the system should be visible or easily retrievable.}
	\item {\textit{Flexibility and efficiency of use}: accelerators (unseen by the novice user)
	      may often speed up the interaction for the expert user such that the system can cater to both
	      inexperienced and experienced users. Allow users to tailor frequent actions.}
	\item {\textit{Aesthetic and minimalist design}: dialogues should not contain information which
	      is irrelevant or rarely needed. Every extra unit of information in a dialogue competes with the
	      relevant units of information and diminishes their relative visibility.}
	\item {\textit{Help users recognize, diagnose, and recover from errors}: error messages should
	      be expressed in plain language, precisely indicate the problem, and constructively suggest a
	      solution.}
	\item {\textit{Help and documentation}: even though it is better if the system can be used without
	      documentation, it may be necessary to provide help and documentation. Any such information should
	      be easy to search, focused on the user’s task, list concrete steps to be carried out, and not be
	      too large. whenever appropriate.}
\end{enumerate}
\mbox{}\\
\noindent
After the expert based evaluation, in the following table, it has been reported that the following
heuristics have been violated:
\begin{figure}[H]
	\centering
	\includegraphics[width=1\textwidth]{images/heuristicEvaluation.png}\\
\end{figure}
\noindent
The severity number identify:
\begin{itemize}
	\item 0 = I don’t agree that this is a usability problem at all
	\item 1 = Cosmetic problem only
	\item 2 = Minor usability problem
	\item 3 = Major usability problem
	\item 4 = Usability catastrophe
\end{itemize}
\mbox{}\\\\\\\\\\
\subsection{Cognitive Walkthrough}

Cognitive Walkthrough is related with the idea of discovering user's cognitive efforts and how much
system's design supports it performing the actions in order to reach its goals.
The idea of method is that provides the expert walks through the system in order to understand if the actions
provided by the system well support the user in doing such task.
The idea of the method is that the experts make use of cognitive psychology in order to understand if the user is well supported, while doing a task,
by the actions provided by the system; more precisely, experts aim to consider the impact of an interaction on the user,
the cognitive process that it requires and which learning problems could arise from it.
The analysis is guided by four predefined questions:
\begin{itemize}
	\item Q1: Is the effect of the action the same as the user’s goal at that point?
	\item Q2: Will users see that the action is available?
	\item Q3: Once users have found the correct action, will they know it is the one they need?
	\item Q4: After the action is taken, will users understand the feedback they get?
\end{itemize}


\paragraph{Task 1 - Search the movie “Harry Potter and the Deathly Hallows - Part 2” to see the movie details.}\mbox{}\\\\
\textit{Action 1}: open the application (assuming that the user is already logged in)\\
\textit{Response 1}: homepage is opened\\
\textit{Action 2}: tap on the search icon\\
\textit{Response 2}: the search page is displayed\\
\textit{Action 3}: tap on the search bar\\
\textit{Response 3}: the keyboard shows up\\
\textit{Action 4}: type “Harry Potter”\\
\textit{Response 4}: each digit is displayed as typed and the system shows the corresponding results\\
\textit{Action 5 – 6 – 7}: [OPTIONAL] tap on filters’ icon, select “most recent” filter and tap on apply\\
\textit{Response 7}: the system shows the results according to the chosen filter\\
\textit{Action 8}: select “Harry Potter and the Deathly Hallows - Part 2” from the list of the results\\
\textit{Response 8}: the systems shows the page of the selected movie containing all the related details
\begin{figure}[H]
	\centering
	\includegraphics[width=0.7\textwidth]{images/mockupSearch.png}\\
\end{figure}
\noindent
Expert evaluation:\\\\
\textit{Action 1}: open the application (assuming that the user is already logged in)\\
\textit{Response 1}: homepage is opened\\\\
\textbf{Q1} - Is the effect of the action the same as the user’s goal at that point?\\
\tab Yes\\
\textbf{Q2} - Will users see that the action is available?\\
\tab Yes\\
\textbf{Q3} - Once users have found the correct action, will they know it is the one they need?\\
\tab Yes\\
\textbf{Q4} - After the action is taken, will users understand the feedback they get?\\
\tab Yes\\\\
\textit{Action 2}: tap on the search icon\\
\textit{Response 2}: the search page is displayed\\\\
\textbf{Q1} - Is the effect of the action the same as the user’s goal at that point?\\
\tab Yes\\
\textbf{Q2} - Will users see that the action is available?\\
\tab Yes\\
\textbf{Q3} - Once users have found the correct action, will they know it is the one they need?\\
\tab Yes\\
\textbf{Q4} - After the action is taken, will users understand the feedback they get?\\
\tab Yes\\\\
\textit{Action 3}: tap on the search bar\\
\textit{Response 3}: the keyboard shows up\\\\
\textbf{Q1} - Is the effect of the action the same as the user’s goal at that point?\\
\tab Yes\\
\textbf{Q2} - Will users see that the action is available?\\
\tab Yes\\
\textbf{Q3} - Once users have found the correct action, will they know it is the one they need?\\
\tab Yes\\
\textbf{Q4} - After the action is taken, will users understand the feedback they get?\\
\tab Yes\\\\
\textit{Action 4}: type “Harry Potter”\\
\textit{Response 4}: each digit is displayed as typed and the system shows the corresponding results\\\\
\textbf{Q1} - Is the effect of the action the same as the user’s goal at that point?\\
\tab Yes\\
\textbf{Q2} - Will users see that the action is available?\\
\tab Yes\\
\textbf{Q3} - Once users have found the correct action, will they know it is the one they need?\\
\tab Yes\\
\textbf{Q4} - After the action is taken, will users understand the feedback they get?\\
\tab Yes\\\\
\textit{Action 5 – 6 – 7}: [OPTIONAL] tap on filters’ icon, select “most recent” filter and tap on apply\\
\textit{Response 7}: the system shows the results according to the chosen filter\\\\
\textbf{Q1} - Is the effect of the action the same as the user’s goal at that point?\\
\tab Yes\\
\textbf{Q2} - Will users see that the action is available?\\
\tab Yes\\
\textbf{Q3} - Once users have found the correct action, will they know it is the one they need?\\
\tab Users with no experience could not recognize the icon\\
\textbf{Q4} - After the action is taken, will users understand the feedback they get?\\
\tab Yes\\\\
\textit{Action 8}: select “Harry Potter and the Deathly Hallows - Part 2” from the list of the results\\
\textit{Response 8}: the systems shows the page of the selected movie containing all the related details\\\\
\textbf{Q1} - Is the effect of the action the same as the user’s goal at that point?\\
\tab Yes\\
\textbf{Q2} - Will users see that the action is available?\\
\tab Yes\\
\textbf{Q3} - Once users have found the correct action, will they know it is the one they need?\\
\tab Yes\\
\textbf{Q4} - After the action is taken, will users understand the feedback they get?\\
\tab Yes\\\\


\paragraph{Task 2 - Add a popular movie to the watchlist.}\mbox{}\\\\
\textit{Action 1}: open the application (assuming that the user is already logged in)\\
\textit{Response 1}: homepage is opened\\
\textit{Action 2}: tap on a movie from “popular movies” section\\
\textit{Response 2}: the movies page is displayed\\
\textit{Action 3}: tap on the “+” button\\
\textit{Response 3}: lists’ popup is displayed\\
\textit{Action 4}: tap on watchlist\\
\textit{Response 4}: the movie is added to watchlist\\\\
\begin{figure}[H]
	\centering
	\includegraphics[width=0.9\textwidth]{images/mockupAdd.png}\\
\end{figure}
\mbox{}\\
\noindent
Expert evaluation:\\\\
\textit{Action 1}: open the application (assuming that the user is already logged in)\\
\textit{Response 1}: homepage is opened\\\\
\textbf{Q1} - Is the effect of the action the same as the user’s goal at that point?\\
\tab Yes\\
\textbf{Q2} - Will users see that the action is available?\\
\tab Yes\\
\textbf{Q3} - Once users have found the correct action, will they know it is the one they need?\\
\tab Yes\\
\textbf{Q4} - After the action is taken, will users understand the feedback they get?\\
\tab Yes\\\\
\textit{Action 2}: tap on a movie from “popular movies” section\\
\textit{Response 2}: the movies page is displayed\\\\
\textbf{Q1} - Is the effect of the action the same as the user’s goal at that point?\\
\tab Yes\\
\textbf{Q2} - Will users see that the action is available?\\
\tab Yes\\
\textbf{Q3} - Once users have found the correct action, will they know it is the one they need?\\
\tab Yes\\
\textbf{Q4} - After the action is taken, will users understand the feedback they get?\\
\tab Yes\\\\
\textit{Action 3}: tap on the “+” button\\
\textit{Response 3}: lists’ popup is displayed\\\\
\textbf{Q1} - Is the effect of the action the same as the user’s goal at that point?\\
\tab It could be not clear that the + button is needed to reach the goal\\
\textbf{Q2} - Will users see that the action is available?\\
\tab Yes\\
\textbf{Q3} - Once users have found the correct action, will they know it is the one they need?\\
\tab Yes\\
\textbf{Q4} - After the action is taken, will users understand the feedback they get?\\
\tab Yes\\\\
\textit{Action 4}: tap on watchlist\\
\textit{Response 4}: the movie is added to watchlist\\\\
\textbf{Q1} - Is the effect of the action the same as the user’s goal at that point?\\
\tab Yes\\
\textbf{Q2} - Will users see that the action is available?\\
\tab Yes\\
\textbf{Q3} - Once users have found the correct action, will they know it is the one they need?\\
\tab Yes\\
\textbf{Q4} - After the action is taken, will users understand the feedback they get?\\
\tab There aren't enough elements to answer\\\\

% Prototype 1 -------------------------------------------------------------------------

\newpage

\section{Prototype 1}

After having received the expert evaluation, we have made some corrections based on it.\\
More specifically, regarding the \textbf{Heuristic Evaluation}, we have made the following corrections:
\paragraph{Login}\mbox{}\\\\
In the login screen we have added “Forgot password?” button in order to let the user restore his password
if he forgot it.
\begin{figure}[H]
	\centering
	\includegraphics[width=0.6\textwidth]{images/prototype1/login.png}\\
	\caption{“Forgot password?” button added to Login page}
\end{figure}
\paragraph{Registration}
\mbox{}\\\\
Here we added a button for the informations about password's formattation rules in order to help the user
understanding how to correctly write the password. Moreover we added another button in order to allow the user
showing/hiding his password.
\begin{figure}[H]
	\centering
	\includegraphics[width=1\textwidth]{images/prototype1/registration.png}\\
	\caption{“Show/hide password” button and information button added to Registration page}
\end{figure}
\paragraph{Movie information}
\mbox{}\\\\
Regarding the movie, in the mockups we had a lot of informations concentrated in a single movie page. For this reason
we have left in the main movie page only the relevant informations, and we have moved in the new “Movie info” page, accessible
by tapping on “View more $\rightarrow$”, all the other informations.\\

\begin{figure}[H]
	\centering
	\includegraphics[width=0.85\textwidth]{images/prototype1/movieInfo.png}\\
	\caption{“Movie” and “Movie info” pages}
\end{figure}

\noindent
The user can arrive in the same page (for example in “Movie” page or “Person” page) from different sections,
so we have supported wayfinding by letting the back button change its label depending on the previous page as
we can see in the example below:
\begin{figure}[H]
	\centering
	\includegraphics[width=0.9\textwidth]{images/prototype1/wayfinding.png}\\
	\caption{Wayfinding support added to all pages}
\end{figure}

\paragraph{Reviews}\mbox{}\\\\
Finally, regarding the page to make a review, we added a popup when the user tap on “Save” button in
order to support error prevention.
\begin{figure}[H]
	\centering
	\includegraphics[width=0.2\textwidth]{images/prototype1/confirmReview.png}\\
	\caption{Confirm review popup added}
\end{figure}

\mbox{}\\
\noindent %TODO vedere
Instead, regarding the \textbf{Cognitive Walkthrough}, we made the following corrections:
\paragraph{Add movie to a list}
\mbox{}\\\\
Regarding the task that allows users to add a movie to a list, we created a single button “Add/remove movie”
instead of the two “+” and “heart icon” buttons, because it could not be clear that they are needed to reach the goal,
as we can see in the Figure \ref{addBtn}.
Also, we added a toast (Figure \ref{toastAddMovie}) that notifies users that the movie is added or removed from a list.

\begin{figure}[H]
	\centering
	\includegraphics[width=0.55\textwidth]{images/prototype1/addBtn.png}\\
	\caption{“+” and “heart icon” buttons modified}
	\label{addBtn}
\end{figure}
\begin{figure}[H]
	\centering
	\includegraphics[width=0.8\textwidth]{images/prototype1/addToList.png}\\
	\caption{Toast added}
	\label{toastAddMovie}
\end{figure}

\paragraph{Search}\mbox{}\\\\
Since not experienced users could not recognize the “Filter” icon, we modified and repositioned the
“Filter” button in order to make it clearer.\\

\begin{figure}[H]
	\centering
	\includegraphics[width=0.6\textwidth]{images/prototype1/filters.png}\\
	\caption{“Filters” button modified}
\end{figure}


% User Based Evaluation -------------------------------------------------------------------------

\newpage

\section{User Based Evaluation}

Once Prototype 1 was up and running based on the corrections related to the expert evaluation, we
proceeded with the user based evaluation. More precisely we used two techniques: \textbf{Think Aloud} and
\textbf{Controlled experiment}.\\ These experiments were conducted over a group of subjects, representative
of the future users, that did not participate in any of the previous phases.
Due to this emergency time that we are living, we exploited the functionality of Zoom that allowed us
to connect with more people in order to have as much results as possible.\\
Our subjects had an age in a range between twenty - thirty years old that fits the age chosen in our user requirements.

\subsection{Think Aloud}

The think aloud is a kind of evaluation based on some simple rules. We chose a group of 10 people and
performed the experiment using these criteria:
\begin{itemize}
	\item We explained to the users who we are and what we were doing.
	\item Each member had to accomplish, individually, the same tasks that are shown below.
	\item We explained that we were testing our application, and not testing them.
	\item The experiment took place remotely, and we provided a demo version of our app "CiakTime" to each person in order to allow them to install it on their smartphones.
	\item While executing the task, each user had to say aloud what he was doing, what he thought it was happening, any doubt, etc., and we were not allowed to help them in any way.
	\item During the experiment, we recorded each person and we took note by pen and paper.
\end{itemize}

\paragraph{Task 1:} \textit{Recently a lot of people are talking about the new Disney's movie called "Luca". So,
	in order to understand the reason behind that, you would like to read its plot. After having noticed that the plot is
	interesting, you would like to add that movie into your "Watchlist" in order to watch it in future.}

\paragraph{Task 2:} \textit{After having watched the movie "Luca", you would like to leave a review about this movie and
	add it to your "Already Watched" list.}

\paragraph{Task 3:} \textit{You really liked the movie "The lord of the rings: the fellowship of the ring" and you would know
	who is the movie director and which other movies he has directed. After having done this search, you would like to go back
	to the search page in order to do other searches.}

\mbox{}\\\\
Regarding \textit{Task 1}, some users, while performing this task, had some difficulties, because they tried to tap on
the three dots at the end of the plot instead the “View more” button, in order to read the entire plot.
Conversely, regarding the other tasks, we noticed that our subjects did not encounter any problem to accomplish them.\\\\\\
In general, we have obtained satisfactory results and therefore we asked the users to freely explore the application in order
to collect additional advices to further improve it.\\\\\\\\
In particular, they gave us some feedback regarding the following features:
\begin{itemize}
	\item {Search between movies and persons should be separated.}
	\item {On person search results they would prefer to have specified whether a person is an actor or
	      a movie director.}
	\item {On the “Movie info” page they would prefer to have specified who is the character that a certain
	      actor plays in that movie.}
\end{itemize}


\subsection{Controlled experiment}

A controlled experiment is an experiment in which all factors are held constant except for one:
the independent variable.\\\\
We used  controlled experiment in order to evaluate two different solutions at the level of graphical interface.
In particular we had two alternatives for the searching page in order to distinguish the search between
movie and person, as it has been suggested by the users during the interviews.
More specifically we realized these two versions in the following ways:
\begin{itemize}
	\item \textbf{Version 1}: With a "tab bar"
	\item \textbf{Version 2}: With a "checkbox"
\end{itemize}

\begin{center}
	\begin{minipage}{0.3\textwidth}
		\begin{figure}[H]
			\centering
			\includegraphics[width=0.8\textwidth]{images/experiment/searchTab.png}\\
			\caption{Version 1}
			\label{fig:version1}
		\end{figure}
	\end{minipage}
	\hspace{0.1\linewidth}
	\begin{minipage}{0.3\textwidth}
		\begin{figure}[H]
			\centering
			\includegraphics[width=0.8\textwidth]{images/experiment/searchCheckBox.png}\\
			\caption{Version 2}
			\label{fig:version2}
		\end{figure}
	\end{minipage}
\end{center}
\mbox{}\\\\
\noindent
These two different alternatives are the independent variables.\\\\
For the experiment, we define the hypothesis and the null hypothesis:
\begin{itemize}
	\item \textbf{Hypothesis}: Users will operate quicker with the Version 1 in Figure \ref{fig:version1} than the Version 2 in Figure \ref{fig:version2}.
	\item \textbf{Null hypothesis}: There will be no speed difference between the two above mentioned versions.
\end{itemize}
Also we have as dependent variable the time that the subjects took to perform the following task:

\paragraph{Task:} \textit{You find the actress Emma Watson very good, so you would like
	get more information about her filmography.}\\

\noindent
For this experiment we used 10 subjects that had an age in a range between twenty - thirty years old, that
fits the age chosen in our user requirements.
We used \textit{within-groups} method, meaning that each subject tested each alternative.
With this method there is the problem of transfer learning; so, in order to avoid this problem,
we have shown to half of the subjects first Version 1 and then Version 2, and to the other
half first Version 2 and then Version 1.

\subsubsection{ANOVA}

Once we have collected the time values from the subjects while performing the task with the two
different versions, we need to analyze them in order to disprove the null hypothesis and to see
if we can confirm our hypothesis.
For doing that, we used ANOVA, a statistical technique that analyzed variance based on the
F-test. We easily computed ANOVA with Microsoft Excel and, in the following Figure (\ref{fig:anova}),
we show the obtained results.

\begin{center}
	\begin{figure}[H]
		\centering
		\includegraphics[width=1\textwidth]{images/anova.png}\\
		\caption{ANOVA result}
		\label{fig:anova}
	\end{figure}
\end{center}

\noindent\mbox{}\\\\
Since F $>$ F crit (7.7 $>$ 4.4), we can reject the null hypothesis and since the average of times in version 1 is less than
the average of times in version 2, we can confirm our hypothesis and state that the first interface (Figure \ref{fig:version1})
is much easier and more immediate to use with respect to the second one (Figure \ref{fig:version2}).\\
After that, we also asked to the subjects a brief comment on the experience and they told us
that they prefer the first interface because is much more intuitive and also because, even though the
second version let them see the results for both movies and persons at the same time, they don't care
about it because, when they perform the search, they have very clear in  mind whether they are searching
for a movie or a person.


% Prototype 2 -------------------------------------------------------------------------

\newpage

\section{Prototype 2}

After the evaluation through user partecipation, we have made some corrections based on it.\\
More specifically, we made the main changes for “Movie” and “Search” pages.

\paragraph{Movie}\mbox{}\\\\
From the \textbf{Think aloud}, we noticed that the users had some difficulties to read the plot because
they tried to tap on the three dots at the end of the plot instead of the “View more” button,
in order to read the entire plot. So we modified the interface in order to allow them to expand the
plot by tapping on “Read more”.

\begin{center}
	\begin{figure}[H]
		\centering
		\includegraphics[width=0.6\textwidth]{images/prototype2/movie.png}\\
		\caption{“Movie” page}
	\end{figure}
\end{center}
\vspace*{-0.5cm}
\noindent
Conversely, from the \textbf{interviews}, some users suggested us to specify who is the character
that a certain actor plays in the movie; so we decided to add this information.
\begin{center}
	\begin{figure}[H]
		\centering
		\includegraphics[width=0.6\textwidth]{images/prototype2/movieInfo.png}\\
		\caption{“Movie info” page}
	\end{figure}
\end{center}
\vspace*{-1.5cm}
\paragraph{Search}\mbox{}\\\\
From the \textbf{interviews}, the subjects told us that they would have preferred to have the possibility
to distinguish search's results between movies and persons instead of obtaining the results of movies and
persons together. Moreover, from \textbf{Controlled experiment}, we noticed that, for the users,
the interface with “tab bar” was more intuitive and effective, so we decided to keep this version
of the interface.
\begin{center}
	\begin{figure}[H]
		\centering
		\includegraphics[width=0.6\textwidth]{images/prototype2/search.png}\\
		\caption{“Search” page modified with a “tab bar”}
	\end{figure}
\end{center}
\vspace*{-0.5cm}
\noindent
Finally, also here, from the \textbf{interviews}, the users told us that they would have preferred
to see whether a searched person was an actor or a movie director.
\begin{center}
	\begin{figure}[H]
		\centering
		\includegraphics[width=0.6\textwidth]{images/prototype2/searchPerson.png}\\
		\caption{Information about persons added}
	\end{figure}
\end{center}

% Final version -------------------------------------------------------------------------

\newpage

\section{Final version}

In this section, we present the final result of our application.

\paragraph{Login}

\begin{center}
	\begin{minipage}[t]{0.31\textwidth}
		\begin{figure}[H]
			\centering
			\includegraphics[width=0.71\textwidth]{images/final/login.png}\\
			\caption{Login page}
		\end{figure}
	\end{minipage}
	\hspace{0.015\linewidth}
	\begin{minipage}[t]{0.31\textwidth}
		\begin{figure}[H]
			\centering
			\includegraphics[width=0.71\textwidth]{images/final/loginGoogle.png}\\
			\caption{Login with Google}
		\end{figure}
	\end{minipage}
	\hspace{0.015\linewidth}
	\begin{minipage}[t]{0.32\textwidth}
		\begin{figure}[H]
			\centering
			\includegraphics[width=0.69\textwidth]{images/final/loginFacebook.png}\\
			\caption{Login with Facebook}
		\end{figure}
	\end{minipage}
\end{center}

\mbox{}\\

\paragraph{Registration}

\begin{center}
	\begin{minipage}[t]{0.31\textwidth}
		\begin{figure}[H]
			\centering
			\includegraphics[width=0.71\textwidth]{images/final/registration.png}\\
			\caption{Registration page}
		\end{figure}
	\end{minipage}
	\hspace{0.015\linewidth}
	\begin{minipage}[t]{0.31\textwidth}
		\begin{figure}[H]
			\centering
			\includegraphics[width=0.71\textwidth]{images/final/infoPass.png}\\
			\caption{Password guidelines}
		\end{figure}
	\end{minipage}
	\hspace{0.015\linewidth}
	\begin{minipage}[t]{0.31\textwidth}
		\begin{figure}[H]
			\centering
			\includegraphics[width=0.71\textwidth]{images/final/infoConfirmPass.png}\\
			\caption{Confirm password guidelines}
		\end{figure}
	\end{minipage}
\end{center}

\paragraph{Home}

\begin{center}
	\begin{figure}[H]
		\centering
		\includegraphics[width=0.23\textwidth]{images/final/home.png}\\
		\caption{Homepage}
	\end{figure}
\end{center}


\paragraph{Search}

\begin{center}
	\begin{minipage}[t]{0.4\textwidth}
		\begin{figure}[H]
			\centering
			\includegraphics[width=0.6\textwidth]{images/final/search.png}\\
			\caption{Search page with “Movie tab” selected}
		\end{figure}
	\end{minipage}
	\hspace{0.015\linewidth}
	\begin{minipage}[t]{0.4\textwidth}
		\begin{figure}[H]
			\centering
			\includegraphics[width=0.6\textwidth]{images/final/searchPeople.png}\\
			\caption{Search page with “Persons tab” selected}
		\end{figure}
	\end{minipage}
\end{center}

\begin{center}
	\begin{minipage}[t]{0.31\textwidth}
		\begin{figure}[H]
			\centering
			\includegraphics[width=0.71\textwidth]{images/final/searchMovie.png}\\
			\caption{Movie search started}
		\end{figure}
	\end{minipage}
	\hspace{0.015\linewidth}
	\begin{minipage}[t]{0.31\textwidth}
		\begin{figure}[H]
			\centering
			\includegraphics[width=0.71\textwidth]{images/final/filters.png}\\
			\caption{Filters}
		\end{figure}
	\end{minipage}
	\hspace{0.015\linewidth}
	\begin{minipage}[t]{0.31\textwidth}
		\begin{figure}[H]
			\centering
			\includegraphics[width=0.71\textwidth]{images/final/searchEmma.png}\\
			\caption{Person search started}
		\end{figure}
	\end{minipage}
\end{center}

\mbox{}\\

\paragraph{User}

\begin{center}
	\begin{figure}[H]
		\centering
		\includegraphics[width=0.23\textwidth]{images/final/user.png}\\
		\caption{User}
	\end{figure}
\end{center}

\mbox{}\\

\paragraph{User settings}

\begin{center}
	\begin{figure}[H]
		\centering
		\includegraphics[width=0.23\textwidth]{images/final/userSetting.png}\\
		\caption{User settings}
	\end{figure}
\end{center}

\begin{center}
	\begin{minipage}[t]{0.46\textwidth}
		\begin{figure}[H]
			\centering
			\includegraphics[width=1\textwidth]{images/final/changeUsernamePass.png}\\
			\caption{Choose new username and password}
		\end{figure}
	\end{minipage}
	\hspace{0.03\linewidth}
	\begin{minipage}[t]{0.46\textwidth}
		\begin{figure}[H]
			\centering
			\includegraphics[width=1\textwidth]{images/final/connectSocial.png}\\
			\caption{Connect account with google and facebook}
		\end{figure}
	\end{minipage}
\end{center}

\mbox{}\\\\\\\\

\paragraph{Movie lists}

\begin{center}
	\begin{minipage}[t]{0.31\textwidth}
		\begin{figure}[H]
			\centering
			\includegraphics[width=0.71\textwidth]{images/final/watchlist.png}\\
			\caption{Watchlist}
		\end{figure}
	\end{minipage}
	\hspace{0.015\linewidth}
	\begin{minipage}[t]{0.31\textwidth}
		\begin{figure}[H]
			\centering
			\includegraphics[width=0.71\textwidth]{images/final/watched.png}\\
			\caption{Already watched list}
		\end{figure}
	\end{minipage}
	\hspace{0.015\linewidth}
	\begin{minipage}[t]{0.31\textwidth}
		\begin{figure}[H]
			\centering
			\includegraphics[width=0.71\textwidth]{images/final/favourite.png}\\
			\caption{Favourite list}
		\end{figure}
	\end{minipage}
\end{center}

\mbox{}\\

\paragraph{Movie}

\begin{center}
	\begin{minipage}[t]{0.4\textwidth}
		\begin{figure}[H]
			\centering
			\includegraphics[width=0.6\textwidth]{images/final/movie.png}\\
			\caption{Movie page}
		\end{figure}
	\end{minipage}
	\hspace{0.015\linewidth}
	\begin{minipage}[t]{0.4\textwidth}
		\begin{figure}[H]
			\centering
			\includegraphics[width=0.6\textwidth]{images/final/movieInfo.png}\\
			\caption{Movie info page}
		\end{figure}
	\end{minipage}
\end{center}

\mbox{}\\

\paragraph{Person}

\begin{center}
	\begin{minipage}[t]{0.4\textwidth}
		\begin{figure}[H]
			\centering
			\includegraphics[width=0.6\textwidth]{images/final/actor.png}\\
			\caption{Person page}
		\end{figure}
	\end{minipage}
	\hspace{0.015\linewidth}
	\begin{minipage}[t]{0.4\textwidth}
		\begin{figure}[H]
			\centering
			\includegraphics[width=0.6\textwidth]{images/final/filmography.png}\\
			\caption{Filmography}
		\end{figure}
	\end{minipage}
\end{center}

\mbox{}\\

\paragraph{Review and comment}

\begin{center}
	\begin{minipage}[t]{0.31\textwidth}
		\begin{figure}[H]
			\centering
			\includegraphics[width=0.71\textwidth]{images/final/reviewPage.png}\\
			\caption{Review page}
		\end{figure}
	\end{minipage}
	\hspace{0.015\linewidth}
	\begin{minipage}[t]{0.31\textwidth}
		\begin{figure}[H]
			\centering
			\includegraphics[width=0.71\textwidth]{images/final/insertReview.png}\\
			\caption{Insert review page}
		\end{figure}
	\end{minipage}
	\hspace{0.015\linewidth}
	\begin{minipage}[t]{0.31\textwidth}
		\begin{figure}[H]
			\centering
			\includegraphics[width=0.71\textwidth]{images/final/comment.png}\\
			\caption{Comment page}
		\end{figure}
	\end{minipage}
\end{center}

% Implementation -------------------------------------------------------------------------

\newpage

\section{Implementation}

For the mockups we used:

\begin{center}
	\begin{figure}[H]
		\centering
		\includegraphics[width=0.28\textwidth]{images/programmiUsati/balsamiq.png}\\
		\caption{Balsamiq Wireframes}
	\end{figure}
\end{center}


\noindent
\mbox{}\\\\
Our application has been developed using:

\begin{center}
	\begin{minipage}{0.45\textwidth}
		\begin{figure}[H]
			\centering
			\includegraphics[width=0.6\textwidth]{images/programmiUsati/flutter.png}\\
			\caption{Flutter framework}
		\end{figure}
	\end{minipage}
	\hspace{0.02\linewidth}
	\begin{minipage}{0.45\textwidth}
		\begin{figure}[H]
			\centering
			\includegraphics[width=0.55\textwidth]{images/programmiUsati/dart.png}\\
			\caption{Dart programming language}
		\end{figure}
	\end{minipage}
\end{center}
\begin{center}
	\begin{figure}[H]
		\centering
		\includegraphics[width=0.2\textwidth]{images/programmiUsati/tmdb.png}\\
		\caption{TMDB APIs}
	\end{figure}
\end{center}


% Conclusion -------------------------------------------------------------------------

\newpage

\section{Conclusion}

After working on this project, it was possible for us to improve a lot our technical skills
in developing mobile applications and interfaces. With respect to other projects, this was the
first time we had a direct contact with the users in order to collect requirements, suggestions,
feedback about general usage concepts. A new way of working has been assimilated.

\subsection{Future implementations}
Some features were not implemented in this project, and have been left to future development.
Here’s a list of some interesting features that could be added:
\begin{itemize}
	\item \textbf{Database}, in order to save persistently information about the users.
	\item \textbf{Movie sections by genre}, in which the users can find movies based on their favourite genres.
	\item \textbf{Suggestions based on movies saved in users lists}.
	\item \textbf{Trailer}, visible in the page of each movie.
\end{itemize}


\end{document}

% COSE UTILI --------------------------------------------------------------------------

%\section*{NOME}
%\subsection*{1.1}
%\setlength{\intextsep}{0pt} --> elimina lo spazio
%\vspace{-3mm}
%\hspace*{0cm}

% Font -------------------------------------

%GRASSETTO: \textbf

% Simboli ---------------------------------

%$\leftarrow$

% Elenco puntato ----------------------

%\begin{itemize}
%\setlength\itemsep{0.01em}
%\item 1
%\item 2
%\end{itemize}

% Graffa grande -----------------------

%\[  
%    \left\{ 
%    \begin{array}{ll} 
%      \mbox{1}
%      \mbox{2}
%    \end{array}
%    \mbox{riga al lato}
%   \right. 
%\]

% Multicolonne --------------------------

% \begin{multicols}{2}
% \columnbreak
% \end{multicols}

% Algoritmi -------------------------------

%\renewcommand{\thealgorithm}{1.\arabic{algorithm}}
%\setcounter{algorithm}{0}
%\begin{algorithm}
%\footnotesize
%\caption{Nome}
%\textbf{Input:} \\
%\textbf{Output:} 

%\begin{algorithmic}[1]
%\STATE 
%\FOR{ = 0 \TO i = n} ---- \ENDFOR
%\IF{} ---- \ELSIF{} ---- \ENDIF
%\RETURN 
%\end{algorithmic}
%\end{algorithm}

% Minipage ------------------------------

%\begin{minipage}[t]{0.5\textwidth}

% queste 3 righe vanno attaccate
%\end{minipage}
%\hspace{0.02\linewidth}
%\begin{minipage}[t]{0.47\textwidth} 

%\begin{minipage}[t]{0.3\textwidth} 
%\end{minipage}

% Proof --------------------------------
%\begin{proof}[\textbf{per cambiare nome}]
%\end{proof}

